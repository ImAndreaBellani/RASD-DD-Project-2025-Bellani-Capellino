\setchapterpreamble[u]{\margintoc}
\chapter{Introduction}
\labch{intro}
	\section{Purpose}
		During their university studies, in order to start entering the workforce, a student might decide to apply for an internship related to their field of study. Similarly, companies offering internships may be interested in finding students that are adequate for them. To facilitate the matching between students and companies, a new platform called \emph{Students and Companies} (S\&C) is to be developed. S\&C allows companies to look for suitable students by publish internship advice on the platform, while students can look for internships that interest them. Moreover, the platform implements recommendation mechanism to help student and companies to find each other. Once the contact is established, S\&C can provide support to the students selection process.
		\subsection{Goals}
			The main goals of the system are:
			
			\quad [G1]\quad students and companies establish contacts for doing internships;
			
			\quad [G2]\quad internships selections can be monitored and supported by the system;
			
			\quad [G3]\quad ongoing internships can be monitored from the system.
	\section{Scope}
		In this section, we are identifying the S\&C domain. In particular, there are two main users categories that interact with the system: \emph{Companies} and \emph{Students}. The companies publish announcements about the internships they want to offer where they specify \emph{projects} that will be carried out and the \emph{terms} of the offer. The system itself informs the companies about the availability of students who may be suitable for their internships (based on their profile).
		
		Students, on the other hand, may use the platform to look for internships and S\&D can also notify them if there are new internships that could meet their interests, but they can still independently search through all the available internships.
		
		[DA METTERE PARTE DELLA SELEZIONE]
		Once a \emph{contact} is established, the student selection process begins and once completed, the system collects feedback and suggestions from both students and companies. Finally, both students and companies can monitor the progress of the internships by providing information on its development and any issues that may arise.
		\subsection{Phenomena}
			\subsubsection{World Phenomena}
			\subsubsection{Shared Phenomena}
				\paragraph{World-controlled Shared Phenomena}
				\paragraph{Machine-controlled Shared Phenomena}
	\section{Definitions, Acronyms and Abbreviations}
		\subsection{Definitions}
		\subsection{Acronyms}
			\begin{itemize}
				\item S\&C: Students\&Companies, the name of the platform;
				\item UML: Unified Modeling Language.
			\end{itemize}
		\subsection{Abbreviations}
			\begin{itemize}
				\item G\_n: Goal number n;
				\item R\_n: Requirement number n;
				\item D\_n: Domain assumption number n;
				\item WP\_n: World Phenomena number n;
				\item SP\_n: Shared Phenomena number n;
				\item CV: Curriculum Vitae;
				\item UC: Use Case.
			\end{itemize}
	\section{Revision history}
	\section{Reference documents}
		The Documents used to deliver the RASD document are the following:
		\begin{itemize}
			\item the Specification of RASD and DD assignment of Software Engineering 2;
			\item the class slides on WeBeep, in particular slides on RE (requirement engineering), scenarios and Use Cases and UML diagrams;
		\end{itemize}
	\section{Document structure}
		\begin{enumerate}
			\item \textbf{INTRODUCTION}: in this section, we provide a brief introduction to the purpose of the platform to be developed, S\&C in this case, focusing in particular on the most important goals which we aim to achieve and on the various phenomena identified;
		\end{enumerate}