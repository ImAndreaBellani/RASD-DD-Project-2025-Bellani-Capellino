%%%%%%%%%%%%%%%%%%%%%%%%%%%%%%%%%%%%%%%%%
% kaobook
% LaTeX Template
% Version 1.3 (December 9, 2021)
%
% This template originates from:
% https://www.LaTeXTemplates.com
%
% For the latest template development version and to make contributions:
% https://github.com/fmarotta/kaobook
%
% Authors:
% Federico Marotta (federicomarotta@mail.com)
% Based on the doctoral thesis of Ken Arroyo Ohori (https://3d.bk.tudelft.nl/ken/en)
% and on the Tufte-LaTeX class.
% Modified for LaTeX Templates by Vel (vel@latextemplates.com)
%
% License:
% CC0 1.0 Universal (see included MANIFEST.md file)
%
%%%%%%%%%%%%%%%%%%%%%%%%%%%%%%%%%%%%%%%%%

%----------------------------------------------------------------------------------------
%	PACKAGES AND OTHER DOCUMENT CONFIGURATIONS
%----------------------------------------------------------------------------------------

\documentclass[
	a4paper, % Page size
	fontsize=10pt, % Base font size
	twoside=false, % Use different layouts for even and odd pages (in particular, if twoside=true, the margin column will be always on the outside)
	%open=any, % If twoside=true, uncomment this to force new chapters to start on any page, not only on right (odd) pages
	%chapterentrydots=true, % Uncomment to output dots from the chapter name to the page number in the table of contents
	numbers=noenddot, % Comment to output dots after chapter numbers; the most common values for this option are: enddot, noenddot and auto (see the KOMAScript documentation for an in-depth explanation)
]{kaobook}
\usepackage{array}
\usepackage{tikz}
\newcolumntype{P}[1]{>{\raggedright\arraybackslash}p{#1}}

% Choose the language
\ifxetexorluatex
	\usepackage{polyglossia}
	\setmainlanguage{english}
\else
	\usepackage[english]{babel} % Load characters and hyphenation
\fi
\usepackage[english=british]{csquotes}	% English quotes

% Load packages for testing
\usepackage{blindtext}
%\usepackage{showframe} % Uncomment to show boxes around the text area, margin, header and footer
%\usepackage{showlabels} % Uncomment to output the content of \label commands to the document where they are used

% Load the bibliography package
\usepackage{kaobiblio}
\addbibresource{main.bib} % Bibliography file

\usepackage{enumitem}
\usepackage{float}
% Load mathematical packages for theorems and related environments
\usepackage[framed=true]{kaotheorems}

% Load the package for hyperreferences
\usepackage{kaorefs}

\graphicspath{{examples/documentation/images/}{images/}} % Paths in which to look for images

\makeindex[columns=3, title=Alphabetical Index, intoc] % Make LaTeX produce the files required to compile the index

\makeglossaries % Make LaTeX produce the files required to compile the glossary
\input{glossary.tex} % Include the glossary definitions

\makenomenclature % Make LaTeX produce the files required to compile the nomenclature

% Reset sidenote counter at chapters
%\counterwithin*{sidenote}{chapter}

%----------------------------------------------------------------------------------------
\usepackage{svg}
\pagelayout{wide} % No margins
\pagelayout{margin} % Restore margins
\setlength{\marginparwidth}{0pt} % Imposta la larghezza della colonna marginale a 0
\setlength{\marginparsep}{0pt}   % Rimuovi lo spazio tra il testo principale e la colonna marginale
\usepackage[utf8]{inputenc}
\usepackage[T1]{fontenc}
\usepackage{hyperref}
\usepackage{ebgaramond}
%\usepackage{fontspec}                                                                                                                       
\addtokomafont{title}{\sffamily}
\addtokomafont{chapter}{\sffamily}
\addtokomafont{section}{\sffamily}
\addtokomafont{subsection}{\sffamily}
\addtokomafont{subsubsection}{\sffamily}
\begin{document}
%\setmainfont{EB Garamond}

%----------------------------------------------------------------------------------------
%	BOOK INFORMATION
%----------------------------------------------------------------------------------------

\titlehead{Requirement Analysis and Specification
	Document}

\title[Requirement Analysis and Specification
Document]{Requirement Analysis and Specification
	Document}
\subtitle{Students\&Companies project Andrea Bellani Alessandro Capellino}

\author[Andrea Bellani, Alessandro Capellino]{Andrea Bellani, Alessandro Capellino}

\date{\today}

\publishers{Politecnico di Milano}

\hypersetup{
	pdftitle={Requirement Analysis and Specification Document},
	pdfauthor={Andrea Bellani},
	pdfsubject={Computer Science and Engineering},
	pdfkeywords={Software Engineering, Computer Science, Engineering, Requirements analysis, Software\&Companies, Politecnico di Milano},
	pdfcreator={LaTeX},
}
%----------------------------------------------------------------------------------------

\frontmatter % Denotes the start of the pre-document content, uses roman numerals

%----------------------------------------------------------------------------------------
%	OPENING PAGE
%----------------------------------------------------------------------------------------

%\makeatletter
%\extratitle{
%	% In the title page, the title is vspaced by 9.5\baselineskip
%	\vspace*{9\baselineskip}
%	\vspace*{\parskip}
%	\begin{center}
%		% In the title page, \huge is set after the komafont for title
%		\usekomafont{title}\huge\@title
%	\end{center}
%}
%\makeatother

%----------------------------------------------------------------------------------------
%	COPYRIGHT PAGE
%----------------------------------------------------------------------------------------

\makeatletter
\uppertitleback{\@titlehead} % Header

\lowertitleback{
	\textbf{Copyright}\\
	\cczero\ 2024, Andrea Bellani Alessandro Capellino – All rights reserved \\
	
	The entire project is available at
	\\\url{https://github.com/ImAndreaBellani/BellaniCapellino}
	
	\textbf{Publisher} \\
	First printed in \today\ by \@publishers
}
\makeatother

%----------------------------------------------------------------------------------------
%	DEDICATION
%----------------------------------------------------------------------------------------

%----------------------------------------------------------------------------------------
%	OUTPUT TITLE PAGE AND PREVIOUS
%----------------------------------------------------------------------------------------

% Note that \maketitle outputs the pages before here

\maketitle

%----------------------------------------------------------------------------------------
%	PREFACE
%----------------------------------------------------------------------------------------

%----------------------------------------------------------------------------------------
%	TABLE OF CONTENTS & LIST OF FIGURES/TABLES
%----------------------------------------------------------------------------------------

\begingroup % Local scope for the following commands

% Define the style for the TOC, LOF, and LOT
%\setstretch{1} % Uncomment to modify line spacing in the ToC
%\hypersetup{linkcolor=blue} % Uncomment to set the colour of links in the ToC
\setlength{\textheight}{230\hscale} % Manually adjust the height of the ToC pages

% Turn on compatibility mode for the etoc package
\etocstandarddisplaystyle % "toc display" as if etoc was not loaded
\etocstandardlines % "toc lines" as if etoc was not loaded

\tableofcontents % Output the table of contents

\listoffigures % Output the list of figures

% Comment both of the following lines to have the LOF and the LOT on different pages
\let\cleardoublepage\bigskip
\let\clearpage\bigskip

\listoftables % Output the list of tables

\endgroup

%----------------------------------------------------------------------------------------
%	MAIN BODY
%----------------------------------------------------------------------------------------

\mainmatter % Denotes the start of the main document content, resets page numbering and uses arabic numbers
\setchapterstyle{kao} % Choose the default chapter heading style

\pagelayout{wide} % No margins
\chapter{Introduction}
\labch{intro}
	\section{Purpose}
		During their university studies, in order to start entering the workforce, a student might decide to apply for an internship related to their field of study. Similarly, companies offering internships may be interested in finding students that are adequate for them. To facilitate the matching between students and companies, a new platform called \emph{Students and Companies} (S\&C) is to be developed. S\&C allows companies to look for suitable students by publish internship advice on the platform, while students can look for internships that interest them. Moreover, the platform implements recommendation mechanism to help student and companies to find each other. Once the contact is established, S\&C can provide support to the students selection process.
		\subsection{Goals}
			The main goals of the system are:
			
			\quad [G1]\quad students and companies establish contacts for doing internships;
			
			\quad [G2]\quad internships selections can be monitored and supported by the system;
			
			\quad [G3]\quad ongoing internships can be monitored from the system.
	\section{Scope}
		In this section, we are identifying the S\&C domain. In particular, there are two main users categories that interact with the system: \emph{Companies} and \emph{Students}. The companies publish announcements about the internships they want to offer where they specify \emph{projects} that will be carried out and the \emph{terms} of the offer. The system itself informs the companies about the availability of students who may be suitable for their internships (based on their profile).
		
		Students, on the other hand, may use the platform to look for internships and S\&D can also notify them if there are new internships that could meet their interests, but they can still independently search through all the available internships.
		
		[DA METTERE PARTE DELLA SELEZIONE]
		Once a \emph{contact} is established, the student selection process begins and once completed, the system collects feedback and suggestions from both students and companies. Finally, both students and companies can monitor the progress of the internships by providing information on its development and any issues that may arise.
		\subsection{Phenomena}
			\subsubsection{World Phenomena}
			\subsubsection{Shared Phenomena}
				\paragraph{World-controlled Shared Phenomena}
				\paragraph{Machine-controlled Shared Phenomena}
	\section{Definitions, Acronyms and Abbreviations}
		\subsection{Definitions}
		\subsection{Acronyms}
			\begin{itemize}
				\item S\&C: Students\&Companies, the name of the platform;
				\item UML: Unified Modeling Language.
			\end{itemize}
		\subsection{Abbreviations}
			\begin{itemize}
				\item G\_n: Goal number n;
				\item R\_n: Requirement number n;
				\item D\_n: Domain assumption number n;
				\item WP\_n: World Phenomena number n;
				\item SP\_n: Shared Phenomena number n;
				\item CV: Curriculum Vitae;
				\item UC: Use Case.
			\end{itemize}
	\section{Revision history}
	\section{Reference documents}
		The Documents used to deliver the RASD document are the following:
		\begin{itemize}
			\item the Specification of RASD and DD assignment of Software Engineering 2;
			\item the class slides on WeBeep, in particular slides on RE (requirement engineering), scenarios and Use Cases and UML diagrams;
		\end{itemize}
	\section{Document structure}
		\begin{enumerate}
			\item \textbf{INTRODUCTION}: in this section, we provide a brief introduction to the purpose of the platform to be developed, S\&C in this case, focusing in particular on the most important goals which we aim to achieve and on the various phenomena identified;
		\end{enumerate}
\setchapterpreamble[u]{\margintoc}
\chapter{Overall description}
\labch{options}

\section{Product perspective}
	\subsection{Scenarios}
		\textbf{Student signs up to S\&C}
		\begin{flushleft}
			Student Bob enters in the system for the first time. On the homepage, he first clicks the \emph{Registration button} and then the \emph{Student Registration button}. To register, Bob fills out a form providing its institutional e-mail (bob.johnson@mail.polimi.it) and password (which will be used for future logins), a brief description of his academic background and specifies whether he would like to receive notifications from the system about future published internships. Finally, Bob uploads his CV by clicking the \emph{Upload CV button}. Now Bob is registered and can search for internships that interest him.
		\end{flushleft}
		\textbf{Company signs up to S\&C}
		\begin{flushleft}
			The company FinestraMI enters the system for the first time. On the homepage, it first clicks the \emph{Registration button} and then the \emph{Company Registration button}. To register, the company fills out a form providing its name, a brief description of its area of expertise and its business area (the market where it operates) and finally its corporate e-mail (info@finestrami.it) and password (which will be used for future logins). FinestraMI also specifies, by selecting the appropriate option, whether it wants to be notified about the availability of students who may be of its interest. Now, FinestraMI is registered and can publish its internships advice.
		\end{flushleft}
		\textbf{Company publishes an internship offer}
		\begin{flushleft}
			The company FinestraMI enters in the system; on the homepage, it clicks the \emph{Login button}. Once logged in, FinestraMI accesses the \emph{Publish New Internship section}. A new internship advice is added by filling out a form where the following information is provided:
			\begin{itemize}
				\item "Window restore" (the intership title);
				\item "The aim of this internship is to give to student to opportunity to repair office windows and..." (a brief description);
				\item "third year bachelor students..." (experience required);
				\item "not suffering from dizziness" (desired skills);
				\item "1. coordination of glass disposal; 2. ..." (main activities the internship involves);
				\item "no paid, canteen tickets available" (terms of the internship);
				\item "\date{22/11/2024}" (advice deadline).
			\end{itemize}
			
			Now the internship advice is visible to students registered on the platform (and also to FinestraMI).
		\end{flushleft}
		\textbf{Student proactively searches for an internship}
		\begin{flushleft}
			Students Bob, Alice and Micheal access to the system by clicking "Login". Each one of them wants to find an internship to apply but each one of them has a different idea of what and where he/she would like to do/be:
			\begin{itemize}
				\item Bob is really interested on doing practice on an handwork but he neither knows a name of a company nor knows which kind of handwork apply for so, he goes to the \emph{View Internships section}, where he can see all the published internships, listed from the most recent to the least recent. The most recent one is "Window restore" by FinestraMI, then he selects it;
				\item Alice has not already decided the kind of internship she wants to apply for but knows many names of companies that operate near her home and so she prefers to go to the \emph{View Companies section}, where she can see all the registered companies and all the internships published by each company. Then she recognized FinestraMI and since she knows that it is expanding, she decides to select it. "Window restore" is the only available advice of FinestraMI but she select it anyways;
				\item Micheal is looking forward to do an internship related to windows restoration, so he uses the search bar to insert "windows restoration" and selects the option "only paid internships", but no internship are found. Then he removes the option and find the internship of FinestraMI. Since it is the only left, he selects it.
			\end{itemize} 
		\end{flushleft}
		\textbf{Student receive a notification about a new internship}
		\begin{flushleft}
			The company CancellaMI (previously registered to the platform) publishes a new internship related to railings maintenance then, Student Bob, who has chosen to be notified by the system when new internships that might be of interest are published, receives an email informing it that a new intership related to his studies is available, since it stated in his CV that after the internship at FinestraMI he became passionate of railings. Bob then logs into the platform and, by going to the \emph{Notification section}, can view the internships offer in more detail.
		\end{flushleft}
		\textbf{Company receives a notification about new possibly interested students}
		\begin{flushleft}
			Company FinestraMI, which has chosen to be notified by the system, receives an email informing it that new students are appealing for its intership "Window Restore" (based on their CVs). FinestraMI then logs into the platform, goes into the \emph{Internship section}, clicks on \emph{Windows restore internship} and by going to the \emph{Notification section} can view the students'profiles and their CVs in more detail.
		\end{flushleft}
		\textbf{Student applies for an internship}
		\begin{flushleft}
			Student Bob wants to apply for the internship "Windows restore". To do so, they log into the system, access the page for "Windows restore" internship and click the \emph{Apply button}. Automatically, the system will send a notification to FinestraMI (the company offering the internship) to inform it that Bob has applied
		\end{flushleft}
		\textbf{The company accepts the application of a student}
		\begin{flushleft}
			Company FinestraMI receives the email regarding student Bob's application for the internship "Window Restore". FinestraMI then logs into the platform, navigates to the \emph{Internships section}, select the \emph{Window Restore Internship}, goes to the \emph{Notification section} and clicks the \emph{Accept Application button} to approve Bob's application.
		\end{flushleft}
		\textbf{The application deadline expires and the selection process is configured}
		\begin{flushleft}
			The administrator of the company FinestraMI notices that the application deadline for the internship advice "Window Restore" (which was previously published on the platform) is now expired and selection process for that internship has not configured yet, so he goes to the designated page and configures:
			\begin{itemize}
				\item two steps (the selection process will be made up of two steps);
				\item a set of metrics to evaluate students ("manual skills" and "knowledge of materials" in this case);
				\item each step is configured as a questionnaire with a series of questions for the students, in this case in particular:
					\begin{enumerate}
						\item first step is test of both open and closed questions regarding knowledge of materials. For closed questions, the platform is also able to automatically check if they are corrected or not (and so, for each closed question, also the scores to assign to each possible answer are inserted into the system). Open questions will be evaluated manually by the company;
						\item second step is an oral exam. Since there are no predefined questions for this step, the company only inserts into the system one open question called "oral exam", scores will be inserted by the company at the end of the exam.
					\end{enumerate}
				\item for each step and for each candidate, the company chooses also the date in which it provides the questionnaire to the candidate.
			\end{itemize}
		\end{flushleft}
		\textbf{The selection process runs}
		\begin{flushleft}
			For the internship advice "Window restore", the company FinestraMI received three applications: Bob, Alice and Micheal. FinestraMI is planning to accept only one student, therefore it chooses to first call Micheal for the first step, since his curriculum impressed more the company. On \date{23/11/2024} Micheal is called and the questionnaire is given to him. His answers are evaluated (automatically for the closed ones and manually for the opened ones) and gets an overall score of 99 out of 100: the company decides to select him and discards any other application for that advice. The company sets for each candidate the right message and the platform notifies them. 
		\end{flushleft}
		\textbf{A student reports a complaint on one of the internship is currently doing}
		\begin{flushleft}
			Today, Alice who is currently enrolled in the internships at the company WeWorkGreat had a problem with the task that was given to her, she asks the helpdesk of the company where she is performing the internship and they ask her to upload a video on the company file sharing platform to show the situation. Alice notices that she can’t upload the video because the maximum uploading size for students is to 10 MB, then she opens Students\&Companies and writes a compliant that states that the file sharing system of WeWorkGreat is only of 10 MB.
		\end{flushleft}
%Considerazioni:
%	i requirements sono stati scritti con questa idea:
%	una compagnia può “proporre” una internship a uno studente 	(manca uno scenario per questo, potrebbe essere aggiunto a quello della notifica). Si veda il requirement “R10602”
%	non ho modellizzato il concetto di “richiesta di candidatura” (ma uno si candida e basta). Se non sei d’accordo lo metto.
% mettere nelle definizioni la differenza tra internship e internship advice

\chapter{Specific requirements}
	\section{External Interface Requirements}
		\subsection{User Interfaces}
		\subsection{Hardware Interfaces}
		\subsection{Software Interfaces}
		\subsection{Communication Interfaces}
	\section{Functional Requirements}
		\subsection{Use-case diagrams}
		\subsection{Use-cases}
		\subsection{Sequence diagrams}
		\subsection{Activity diagrams}
		\subsection{Requirements mapping}
			\begin{table} [h!]
				\centering
				\begin{tabular}{ || P{7.5cm} P{7.5cm} || }
					\hline
						\multicolumn{2}{||P{15cm}||}{[G1] students and companies establish contacts for doing internships} \\ [0.5ex]
					\hline
					[R10101] the system allows students to sign up to the platform with their institutional mails & [D10101] students upload their CV in Europass format \\
					
					[R10102] the system allows a student to set up whether he/she wants to be notified of the presence of internship advice that might interest him/her & [D10102] information on a student CV do not contradict each other \\
					
					[R10103] the system allows students upload their CV to the platform & [D10302] information companies insert in internship advice do not contradict each other \\
					
					[R10104] the system allows students to publish on their profile a brief description of themselves & \\
					
					[R10201] the system allows companies to sign up to the platform with their company address & \\
					
					[R10202] the system allows companies to insert the main information regarding their business area and area of expertise & \\
					
					[R10203] the system allows a company to set up whether it wants to be notified of the presence of students that might be interested to its internship advice & \\
					
					[R10301] the system allows companies to publish internship advice where they specify the main information regarding the internship (brief description, experience required, desired skills, main activities involved and the terms) and the submission deadline & \\
					
					[R10401] the system allows students to search internships advice by name (and also to see the complete list of available advice). The system shall act as a search engine to present also the names of the advice that are similar to the searched one & \\
					
					[R10402] the system allows students to search companies by name (and also to see the complete list of registered companies) and then access to their profile & \\
					
					[R10403] the system allows students to filter the results they searched (e.g. "only paid internships", "only companies located in Lombardy") & \\
					
					[R10501] when the system recognizes that a new internship advice that might interest a student (that allowed the notification option) is published it notifies that student by sending him an e-mail (to your address) & \\
					
					[R10601] when the system recognizes that a student has a profile that would fit an internship advice, the company that published the advice is notified & \\
					
					[R10602] when a company opens a student profile, it can propose to him to apply for one of its internships & \\
					
					[R10701] the system allows students to apply for any internship advice which deadline has not expired & \\
					
					[R10702] when a student applies for an internship, the related company is notified by the system & \\ [1ex]
					\hline
				\end{tabular}
				\caption{Requirements mapping for goal G1}
				\label {table:1}
			\end{table}
			\begin{table} [h!]
				\centering
				\begin{tabular}{ || P{7.5cm} P{7.5cm} || }
					\hline
					\multicolumn{2}{||P{15cm}||}{[G2] internships selections can be monitored and supported by the system} \\ [0.5ex]
					\hline
					[R20101] when the deadline for an internship advice is expired, the system allows the company to set up the selection process by specifying for each step, the relative questionnaire (with metrics for each question) and the date in which provide it to a student (dates may differ between different students) & \\
					
					[R20201] the system automatically calculates the scores of questionnaire closed answers & \\
					
					[R20202] the system allows companies to manually insert scores for questionnaire open answers & \\
					
					[R20203] the system allows companies to visualize and compare selections scores & \\
					
					[R20204] in any selection phase, the system allows companies to discard a student currently involved in the selection process (discarded students are removed by the selection process) & \\
					
					[R20205] in any selection phase, the system allows companies to accept a student currently involved in the selection process (accepted students are removed by the selection process) & \\
					
					[R20206] the system allows companies to write a personalized message to communicate the result of a selection & \\ [1ex]
					\hline
				\end{tabular}
				\caption{Requirements mapping for goal G2}
				\label {table:1}
			\end{table}
			\begin{table} [h!]
				\centering
				\begin{tabular}{ || P{7.5cm} P{7.5cm} || }
					\hline
					\multicolumn{2}{||P{15cm}||}{[G3] ongoing internships can be monitored from the system} \\ [0.5ex]
					\hline
					[R30101] the system allows students and companies to consult the internships (ongoing or finished) & \\
					
					[R30102] the system allows students and companies to report complaints on the internships they are involved in & \\ [1ex]
					\hline
				\end{tabular}
				\caption{Requirements mapping for goal G3}
				\label {table:1}
			\end{table}
	\section{Performance Requirements}
	\section{Design Constraints}
		\subsection{Standards compliance}
		\subsection{Hardware limitations}
		\subsection{Other constraints}
	\section{Software System Attributes}
		\subsection{Reliability}
		\subsection{Availability}
		\subsection{Security}
		\subsection{Maintainability}
		\subsection{Portability}
\chapter{Requirements traceability}
	Here we detail how each requirement stated in the RASD is effectively ensured through the components:

	\section{General requirements}
			\begin{table}[H]
				\begin{tabular}{ | m{2.6cm} | m{9cm} | } 
					\hline
					\textbf {Requirement} & \textbf{Component(s)} \\
					\hline
						R00000 & NotificationGenerator, SES API \\
					\hline
				\end{tabular}
				\caption{General requirements traceability table}
			\end{table}
	\section{Requirements related to goal G1}
			\begin{table}[H]
				\begin{tabular}{ | m{2.6cm} | m{9cm} | } 
					\hline
					\textbf {Requirement} & \textbf{Component(s)} \\
					\hline
						R10101 & AuthenticationManager, ProfileManager\\
					\hline
						R10102 & AuthenticationManager, ProfileManager \\
					\hline
						R10103 & AuthenticationManager, ProfileManager, MongoDB API (for CV)\\
					\hline
						R10104 & AuthenticationManager, ProfileManager\\
					\hline
						R10105 & AuthenticationManager \\
					\hline
						R101016 & ProfileManager, MongoDB API (for CV) \\
					\hline
						R10107 & ProfileManager \\
					\hline
						R10201 & AuthenticationManager, ProfileManager \\
					\hline
						R10202 & AuthenticationManager, ProfileManager \\
					\hline
						R10203 & AuthenticationManager, ProfileManager \\
					\hline
						R10204 & AuthenticationManager \\
					\hline
						R10205 & ProfileManager \\
					\hline
						R10301 & AdvicePublisher \\
					\hline
						R10302 & AdvicePublisher \\
					\hline
						R10401 & AdvicePresenter \\
					\hline
						R10402 & ProfilePresenter \\
					\hline
						R10403 & AdvicePresenter \\
					\hline
						R10404 & AdvicePresenter \\
					\hline
						R10501 & RecommendationInterface, RecommendationAnalyzer, RecommendationPresenter, NotificationGenerator\\
					\hline
						R10601 & RecommendationInterface, RecommendationAnalyzer, RecommendationPresenter, NotificationGenerator\\
					\hline
						R10701 & ApplicationManager \\
					\hline
						R10702 & NotificationGenerator \\
					\hline
						R10801 & ApplicationManager \\
					\hline
						R10901 & ApplicationManager, NotificationGenerator \\
					\hline
						R11001 & ApplicationManager \\
					\hline
						R11002 & ApplicationManager \\
					\hline
						R12001 & NotificationGenerator, NotificationPresenter, FeedbackManager \\
					\hline
						R12002 & FeedbackManager \\
					\hline 
						R11301 & InternshipManager, FeedbackManager \\
				\end{tabular}
				\caption{requirements related to goal G1 table}
			\end{table}
	\section{Requirements related to goal G2}
			\begin{table}[H]
				\begin{tabular}{ | m{2.6cm} | m{9cm} | } 
					\hline
					\textbf {Requirement} & \textbf{Component(s)} \\
					\hline
						R20101 & SPInitializer, MongoDB API \\
					\hline
						R20102 & SPInitializer \\
					\hline
						R20201 & SPPresenter, NotificationGenerator \\
					\hline
						R20202 & SPManager, MongoDB API \\
					\hline
						R20203 & SPManager \\
					\hline
						R20204 & SPManager, MongoDB API \\
					\hline
						R20205 & SPPresenter \\
					\hline
						R20206 & SPManager \\
					\hline
						R20207 & SPManager \\
					\hline
						R20209 & SPManager, NotificationGenerator \\
					\hline
						R20210 & SPManager \\
					\hline
				\end{tabular}
				\caption{Requirements related to goal G2 traceability table}
			\end{table}
	\section{Requirements related to goal G3}
			\begin{table}[H]
				\begin{tabular}{ | m{2.6cm} | m{9cm} | } 
					\hline
					\textbf {Requirement} & \textbf{Component(s)} \\
										\hline
						R30101 & InternshipManager \\
					\hline
						R30102 & InternshipManager \\
					\hline
						R30103 & InternshipManager \\
					\hline
						R30104 & NotificationGenerator, InternshipManager \\
					\hline
						R30104 & InternshipManager \\
					\hline
				\end{tabular}
				\caption{Requirements related to goal G3 traceability table}
			\end{table}
\chapter{Effort Spent}
	\begin{center}
		\begin{table}[H]
			\begin{tabular}{ | m{3.2cm} | m{1cm}| m{4cm} | m{1.5cm}| m{4cm} | } 
				\hline
					&  \multicolumn{2}{c|}{ Andrea} & \multicolumn{2}{c|}{ Alessandro} \\ 
				\hline
					\textbf{Week} & \textbf{Hours}   & \textbf{Category} & \textbf{Hours}       & \textbf{Category} \\
				\hline
					21/10/2024 - 27/10/2024 & 6 & Resume course notes and assignment analysis & 3 & Assignment analysis\\
				\hline
					28/10/2024 - 03/11/2024 & 0.5 & First goal definition & 2.5 & Deeper assignment analysis and first scenarios \\
				\hline
					04/11/2024 - 10/11/2024 & 1.5 & First requirements definition and first high-level class diagram & 3 & RASD introduction and first phenomena definition\\
				\hline
					11/11/2024 - 17/11/2024 & 0 & - & 2 & Deeper scenarios definition \\
				\hline
					18/11/2024 - 24/11/2024 & 12 & First Alloy modeling, requirements deeper analysis, selection process scenarios & 4 & Phenomena deeper analysis, scenarios adjustments\\
				\hline
					25/11/2024 - 01/12/2024 & 1.5 & Requirements and class diagram adjustments & 3 & Use-case initial definition \\
				\hline
					02/12/2024 - 08/12/2024 & 2 & Scenarios and requirements completion and non-functional requirements definition & 4 & Use-case deeper definition and non-functional requirements definition\\
				\hline
					09/12/2024 - 15/12/2024 & 8 & Alloy completion and product-functions & 6 & Use-cases completion, sequence and state diagrams definition\\
				\hline
					16/12/2024 - 22/12/2024 & 5 & Final overview, Effort Spent compilation and User Interfaces & 9 & Final overview and Sequence, state, activity, use-case diagrams completion and User Interfaces\\
				\hline
					\textbf{TOTAL} & \textbf{36.5} & & \textbf{36.5} & \\
				\hline
			\end{tabular}
			\caption{Effort spent overview}
		\end{table}
	\end{center}
\chapter{Effort Spent}
\begin{center}
	\begin{table}[H]
		\begin{tabular}{ | m{3.2cm} | m{1cm}| m{4cm} | m{1.5cm}| m{4cm} | } 
			\hline
			&  \multicolumn{2}{c|}{ Andrea} & \multicolumn{2}{c|}{ Alessandro} \\ 
			\hline
			\textbf{Week} & \textbf{Hours}   & \textbf{Category} & \textbf{Hours}       & \textbf{Category} \\
			\hline
			23/12/2024 - 29/12/2024 & 16 & User interface and high-level structure & 4 & high-level structure\\
			\hline
			30/12/2024 - 05/01/2025 & 2 & Data logic model & 12 & Introduction, data logic model, sequence diagrams\\
			\hline
			06/01/2025 - 07/01/2025 & 8 & Final revision, requirements mapping, component view, deployment view, effort spent and software used compilation & 8 & Final revision, requirements mapping, integration and test plan, component view, integration and test plan\\
			\hline
			\textbf{TOTAL} & \textbf{26} & & \textbf{24} & \\
			\hline
		\end{tabular}
		\caption{Effort spent overview}
	\end{table}
\end{center}

%----------------------------------------------------------------------------------------
%	INDEX
%----------------------------------------------------------------------------------------

% The index needs to be compiled on the command line with 'makeindex main' from the template directory

\printindex % Output the index

%----------------------------------------------------------------------------------------
%	BACK COVER
%----------------------------------------------------------------------------------------

% If you have a PDF/image file that you want to use as a back cover, uncomment the following lines

%\clearpage
%\thispagestyle{empty}
%\null%
%\clearpage
%\includepdf{cover-back.pdf}

%----------------------------------------------------------------------------------------

\end{document}
